\documentclass{article}
\usepackage{scrextend}
\usepackage[english, russian]{babel}
\title{Математический анализ\\*Типовой расчет №1\\*Неопределенные и определенные интегралы\\*ФИТиП ИС\\*1 курс 3 модуль}
\date{02.2020 - 03.2020}
\author{Кулешова Екатерина Дмитриевна \\* M3103}

\usepackage{amsmath, latexsym, commath, amssymb}
\usepackage[left=2cm,right=2cm,top=2cm,bottom=2cm,bindingoffset=0cm]{geometry}
\usepackage{graphicx}
\graphicspath{{C:/Users/ked/ITMO/TeXdocuments/matanSem2/}}
\DeclareGraphicsExtensions{.pdf,.png,.jpg}

\newcommand{\dreplace}[2]{\textrm{Замена: }\left[
	\begin{array}{c} #1\\#2
	\end{array}\right]}

\newcommand{\treplace}[3]{\textrm{Замена: }\left[
	\begin{array}{c} #1\\#2\\#3
	\end{array}\right]}

\newcommand{\freplace}[4]{\textrm{Интегрируем по частям: }\left[
	\begin{array}{r|l} #1 & #2 \\ #3 & #4
	\end{array}\right]}

\renewcommand{\tan}{tg}
\renewcommand{\cot}{ctg}
\renewcommand{\arctan}{arctg}
\renewcommand{\cosh}{ch}
\renewcommand{\sinh}{sh}


\begin{document}
	\maketitle
	\newpage
	\section{Найти интегралы}
		\subsection*{\ \ \ \ \ \ a) \ $\int e^{5x - 1} \dif x = (*)$}
			Используем теорему о замене переменной в неопределенном интеграле, чтобы получить табличный интеграл:
			\begin{multline*}
				\dreplace{t = 5x - 1}{\dif x = \dfrac{\dif t}5} => (*) = \dfrac 1 5 \int e^t \dif t = \dfrac 1 5 (e^t + C) = \dfrac{e^{5x - 1}}5  + C
			\end{multline*}

		\subsection*{\ \ \ \ \ \ б) \ $\int \dfrac{\cos x}{\sqrt[3]{\sin^2 x}} \dif x = (*)$}
			Используем теорему о замене переменной в неопределенном интеграле, чтобы получить табличный интеграл:
			\begin{multline*}
				\dreplace{t = \sin x}{\dif t = \cos x \dif x} => (*) =  \int \dfrac{\dif t}{\sqrt[3]{t^2}} = 3 \sqrt[3] t + C = 3 \sqrt[3]{\sin x} + C
			\end{multline*}
			
		\subsection*{\ \ \ \ \ \ в) \ $\int \dfrac{x - 3}{\sqrt{3 - 2x - x^2}} \dif x = (*)$}
			Преобразуем исходное выражение, ставим знаменатель под знак интеграла, в оставшейся части выделим полные квадраты, благодаря чему получаем два табличных интеграла:
			\begin{multline*}
				(*) = \dfrac 1 2 \int \dfrac{2x - 6}{\sqrt{3 - 2x - x^2}} \dif x = \dfrac 1 2 \int \left(\dfrac{2x + 2}{\sqrt{3 - 2x - x^2}} - \dfrac 8{\sqrt{3 - 2x - x^2}}\right) \dif x =\\= -\dfrac 1 2 \int \dfrac{-2x - 2}{\sqrt{3 - 2x - x^2}} \dif x - 4 \int \dfrac{\dif x}{\sqrt{3 - 2x - x^2}} = -\dfrac 1 2 \int \dfrac{-2x - 2}{\sqrt{3 - 2x - x^2}} \dif x - 4 \int \dfrac{\dif x}{\sqrt{4 - \left(x + 1\right)^2}} =\\= -\dfrac 1 2 \int \dfrac{\dif \left(3 - 2x - x^2\right)}{\sqrt{3 - 2x - x^2}} - 4 \int \dfrac{\dif \left(x + 1\right)}{\sqrt{4 - \left(x + 1\right)^2}} = - \sqrt{3 - 2x - x^2} - 4 \arcsin{\dfrac{x + 1}2} + C\\ 
			\end{multline*}
		
		\section{Найти интегралы с помощью интегрирования по чаcтям и замены переменной}
			\subsection*{\ \ \ \ \ \ a) $\int x \ \log_3{\left(1 - x\right)} \dif x = (*)$}
				Под знаком интеграла стоит произведение полинома на трансцендентную функцию - это типовая ситуация. Избавляемся от трансцендентности, интегрируя по частям, где за u берем эту функцию, ее производная будет уже алгебраической функцией:
				\begin{multline*}
					\freplace{u = \log_3{\left(1 - x\right)}}{\dif u = -\dfrac{\dif x}{\left(1 - x\right) \ln 3}}{v = \dfrac{x^2}2}{\dif v =  x \ \dif x} => (*) = \dfrac{x^2}2 \ \log_3{\left(1 - x\right)} - \left(\int \dfrac{-x}{\left(1 - x\right) \ln 3} \dif x \right) =\\= \dfrac{x^2}2 \ \log_3{\left(1 - x\right)} - \dfrac 1{\ln 3} \int \left(1 - \dfrac 1{1 - x}\right) \dif x = \dfrac{x^2}2 \ \log_3{\left(1 - x\right)} - \dfrac {x}{\ln 3} - \dfrac 1{\ln 3} \int \dfrac { \dif\left(1 - x\right)}{1 - x} =\\= \dfrac{x^2}2 \ \dfrac{\ln{\left(1 - x\right)}}{\ln 3} - \dfrac {x}{\ln 3} - \dfrac {\ln{\left|1 - x\right|}}{\ln 3} + C = \dfrac{x^2}2 \ \log_3{\left(1 - x\right)} - \log_3{\left|1 - x\right|} - \dfrac {x}{\ln 3} + C =\\= \dfrac{\dfrac{2x^2 }2\log_3{\left(1 - x\right)} - 2\log_3{\left|1 - x\right|}}{2} - \dfrac {x}{\ln 3} + C = \dfrac{\dfrac{x^2 }2\log_3{\left(1 - x\right)^2} - \log_3{\left(1 - x\right)^2}}{2} - \dfrac {x}{\ln 3} + C = \left(\dfrac{x^2}4 - \dfrac 1 2\right) \log_3{\left(1 - x\right)^2} - \dfrac {x}{\ln 3} + C
				\end{multline*}
				
			\subsection*{\ \ \ \ \ \ б) $\int \dfrac{\sqrt x}{1 + \sqrt x} \dif x = (*)$}
				Избавимся от иррациональности, заменяя знаменатель дроби, после чего разложим дробь на слагаемые - табличные интегралы:
				\begin{multline*}
					\treplace{t = 1 + \sqrt x, \ \sqrt x = t - 1}{\dif t = \dfrac{\dif x}{2 \sqrt x}}{\dif x = 2 \sqrt x \ \dif t = 2\left(t - 1\right) \ \dif t} => (*) = 2 \int \dfrac{\left(t - 1\right)^2}{t} \dif t = 2 \int \dfrac{t^2 - 2t + 1}{t} \dif t = 2 \int \left(t - 2 + \dfrac 1 t\right) \dif t =\\= 2 \int t \dif t - 4 \int \dif t + 2\int \dfrac{\dif t}{t} = t^2 - 4t + 2 \ln t	+ C = {\left(1 + \sqrt x\right)}^2 - 4\left(1 + \sqrt x\right) + 2 \ln{\left(1 + \sqrt x\right)} + C =\\= 1 + 2\sqrt x + x - 4 - 4\sqrt x + 2 \ln{\left(1 + \sqrt x\right)} + C = 2 \ln{\left(1 + \sqrt x\right)} + x - 2\sqrt x + C 
				\end{multline*}
				
		\section{Найти неопределенные интегралы от рациональных функций}
			\subsection*{\ \ \ \ \ \ $\int \dfrac{x^3 + 9x^2 + 21x + 21}{\left(x^2 + 3\right)\left(x + 3\right)^2} \dif x = (*)$}
				Проверим, является ли данная дробь правильной
				\begin{multline*}
					(*) = \dfrac{x^3 + 9x^2 + 21x + 21}{\left(x^2 + 3\right)\left(x^2 + 6x + 9\right)} = \dfrac{x^3 + 9x^2 + 21x + 21}{x^4 + 6x^3 + 9x^2 + 3x^2 + 18x + 27} = \dfrac{x^3 + 9x^2 + 21x + 21}{x^4 + 6x^3 + 12x^2 + 18x + 27}
				\end{multline*}
				Степень знаменателя больше степени числителя, значит дробь правильная. Разложим ее на простейшие слагаемые методом неопределенный коэфициентов (по теореме о разложении):
				\begin{multline*}
					\dfrac{x^3 + 9x^2 + 21x + 21}{\left(x^2 + 3\right)\left(x + 3\right)^2} = \dfrac{Ax + B}{x^2 + 3} + \dfrac{C}{x + 3} + \dfrac{D}{\left(x + 3\right)^2} = \dfrac{\left(Ax + B\right)\left(x + 3\right)^2 + C\left(x^2 + 3\right)\left(x + 3\right) + D\left(x^2 + 3\right)}{\left(x^2 + 3\right)\left(x + 3\right)^2} =>\\ x^3 + 9x^2 + 21x + 21 = \left(Ax + B\right)\left(x + 3\right)^2 + C\left(x^2 + 3\right)\left(x + 3\right) + D\left(x^2 + 3\right) =\\= A\left(x^3 + 6x^2 + 9x\right) + B\left(x^2 + 6x + 9\right) + C\left(x^3 + 3x^2 + 3x + 9\right) + D\left(x^2 + 3\right)\\
					\textrm{Коэффициенты перед степенями x удовлетворяют следующим уравнениям:}\\ \begin{array}{l|l} x^0 & 21 = 9B + 9C + 3D \\ x^1 & 21 = 9A + 6B + 3C \\ x^2 & 9 = 6A + B + 3C + D \\ x^3 & 1 = A + C\end{array} =>\\\textrm{Решим данную систему методом Гаусса, и найдем коэффициенты после обратного прохода:}\\\left(\begin{array}{llll|l}1 & 0 & 1 & 0 & 1 \\ 6 & 1 & 3 & 1 & 9 \\ 3 & 2 & 1 & 0 & 7 \\ 0 & 3 & 3 & 1 & 7\end{array}\right) \thicksim \left(\begin{array}{llll|l}1 & 0 & 1 & 0 & 1 \\ 0 & 1 & -3 & 1 & 3 \\ 0 & 2 & -2 & 0 & 4 \\ 0 & 3 & 3 & 1 & 7\end{array}\right) \thicksim \left(\begin{array}{llll|l}1 & 0 & 1 & 0 & 1 \\ 0 & 1 & -3 & 1 & 3 \\ 0 & 0 & 4 & -2 & -2 \\ 0 & 0 & 12 & -2 & -2\end{array}\right) \thicksim \left(\begin{array}{llll|l}1 & 0 & 1 & 0 & 1 \\ 0 & 1 & -3 & 1 & 3 \\ 0 & 0 & 2 & -1 & -1 \\ 0 & 0 & 0 & -8 & -8\end{array}\right) \thicksim \\ \thicksim \left(\begin{array}{llll|l}1 & 0 & 1 & 0 & 1 \\ 0 & 1 & -3 & 0 & 2 \\ 0 & 0 & 2 & 0 & 0 \\ 0 & 0 & 0 & 1 & 1\end{array}\right) \thicksim \left(\begin{array}{llll|l}1 & 0 & 0 & 0 & 1 \\ 0 & 1 & 0 & 0 & 2 \\ 0 & 0 & 1 & 0 & 0 \\ 0 & 0 & 0 & 1 & 1\end{array}\right) => A = 1, \ B = 2, \ C = 0, \ D = 1 \\(*) = \int \left(\dfrac{x + 2}{x^2 + 3} + \dfrac 1{\left(x + 3\right)^2}\right) \dif x = \dfrac 1 2 \int \dfrac{2x + 4}{x^2 + 3} \dif x + \int \dfrac{\dif x}{\left(x + 3\right)^2} = \dfrac 1 2 \left(\int \dfrac{\dif \left(x^2 + 3\right)}{x^2 + 3} + 4 \int \dfrac{\dif x}{x^2 + 3} \right) + \int \dfrac{\dif x}{\left(x + 3\right)^2} =\\= \dfrac 1 2 \int \dfrac{\dif \left(x^2 + 3\right)}{x^2 + 3} + 2 \int \dfrac{\dif x}{x^2 + 3} + \int \dfrac{\dif \left(x + 3\right)}{\left(x + 3\right)^2} = \dfrac 1 2 \ln{\left(x^2 + 3\right)} + \dfrac 2{\sqrt 3} \arctan{\left(\dfrac x{\sqrt 3}\right)} - \dfrac 1{x + 3} + C
				\end{multline*}
			
		\section{Найти неопределенные интегралы}
			\subsection*{\ \ \ \ \ \ $\int \dfrac{\sin x}{\left(1 - \sin x + \cos x\right)^2} \dif x = (*)$}
				Используя универсальную тригонометрическую подтановку, сведем к интегралу от рациональной функции, которую потом разложим на простейшие слагаемые и получим табличные интегралы:
				\begin{multline*}
					\treplace{\tan\left(\dfrac x 2\right) = t}{dx = \dfrac{2 \dif t}{1 + t^2}}{\sin x = \dfrac{2t}{1 + t^2}, \ \cos x = \dfrac{1 - t^2}{1 + t^2}} => (*) = \int \dfrac{\dfrac{2t}{1 + t^2}}{\left(1 - \dfrac{2t}{1 + t^2} + \dfrac{1 - t^2}{1 + t^2}\right)^2} \dfrac{2 \dif t}{1 + t^2} =\\\\= 4\int \dfrac{t \left(1 + t^2\right)}{\left(1 + t^2 - 2t + 1 - t^2\right)\left(1 + t^2\right)^2} \dif t = 4\int \dfrac{t \dif t}{\left(2 - 2t\right)\left(1 + t^2\right)} = 2 \int \dfrac{t \dif t}{\left(1 - t\right)\left(1 + t^2\right)}
				\end{multline*}
				Степень знаменателя больше степени числителя, значит дробь правильная. Разложим ее на простейшие слагаемые методом неопределенный коэфициентов (по теореме о разложении):
				\begin{multline*}
					\dfrac{2t}{\left(1 - t\right)\left(1 + t^2\right)} = \dfrac A{1 - t} + \dfrac{Bt + C}{1 + t^2} = \dfrac{A\left(1 + t^2\right) + \left(Bt + C\right)\left(1 - t\right)}{\left(1 - t\right)\left(1 + t^2\right)} = \dfrac{A\left(1 + t^2\right) + B\left(t -t^2\right) + C\left(1 - t\right)}{\left(1 - t\right)\left(1 + t^2\right)}\\
					\textrm{Коэффициенты перед степенями t удовлетворяют следующим уравнениям:}\\ \begin{array}{l|l} x^0 & 0 = A + C\\ x^1 & 2 = B - C \\ x^2 & 0 = A - B\end{array} => \ A = 1, \ B = 1, \ C = -1\\(*) =  \int \left(\dfrac 1{1 - t} + \dfrac{t - 1}{1 + t^2}\right) \dif t = -\int \dfrac{\dif \left(1 - t\right)}{1 - t} + \dfrac 1 2 \int \dfrac{2t - 2}{1 + t^2} \dif t = -\int \dfrac{\dif \left(1 - t\right)}{1 - t} + \dfrac 1 2 \int \dfrac{\dif \left(1 + t^2\right)}{1 + t^2}  - \int \dfrac{\dif t}{1 + t^2} =\\= -\ln \left|1 - t\right| + \dfrac 1 2 \ln \left|1 + t^2\right| - \arctan \left(t\right) + C = \dfrac 1 2 \ln \left(\dfrac{1 + \tan^2 \dfrac x 2}{\left(1 - \tan \dfrac x 2 \right)^2}\right) - \dfrac x 2 + C = \dfrac 1 2 \ln \left( 1 + \dfrac{2\tan\dfrac x 2}{\left(1 - \tan \dfrac x 2 \right)^2}\right) - \dfrac x 2 + C
				\end{multline*}
				
		\section{Вычислить площадь фигуры, ограниченной линиями}	
			\subsection*{\ \ \ \ \ \ $x = \sqrt{e^y - 1}, \ x = 0, \ y = \ln 2$}
				Если область D ограничена сверху кривой $x = \Phi\left(y\right)$, а снизу кривой $x = \phi\left(y\right)$, причём $\phi\left(y\right) \le \Phi\left(y\right), \ y \in [a, b]$, то площадь области  можно вычислить по формуле: $S = \int_a^b \left(\Phi\left(y\right) - \phi\left(y\right)\right)$
				\begin{figure}
					\centering
					\includegraphics{task_5}
				\end{figure}
				Судя по графику, ограниченная область задается неравенствами: $0 \le x \le \sqrt{e^y - 1}, \ 0 \le y \le \ln 2$ (в данном случае удобно интегрировать по y). Т.к $x = \phi\left(y\right) = 0$, то вычитаемое зануляется, и его можно не учитывать:
				\begin{multline*}
					 => \  S = \int_0^{\ln 2}{\sqrt{e^y - 1}} \dif y\\\treplace{t = \sqrt{e^y - 1}, \ e^y = t^2 + 1}{\dif t = \dfrac{e^y}{2\sqrt{e^y - 1}} \dif y = \dfrac{t^2 + 1}{2t} \dif y}{\dif y = \dfrac{2t \dif t}{t^2 + 1}} \ \ \ \ \begin{array}{l|l} y & t \\ -- & ----------- \\ 0 & 0 \\ \ln 2 & \sqrt{e^{\ln 2} - 1} = \sqrt{2 - 1} = 1\end{array}\\ => S = 2\int_0^1 \dfrac{t^2}{t^2 + 1} \dif t = 2\int_0^1 \left(\dfrac{t^2 + 1}{t^2 + 1} - \dfrac{1}{t^2 + 1}\right) \dif t = 2\int_0^1 \dif t - 2\int\dfrac{\dif t}{t^2 + 1} = 2\left(t\right)|_0^1 - 2\arctan\left(t\right)|_0^1 =\\= 2\left( 1 - 0 - \arctan 1 + \arctan 0\right) = 2\left( 1 - 0 - \dfrac{\pi}{4} + 0\right) = 2 - \dfrac{\pi}2
				\end{multline*}
				
		\section{Вычислить длины дуг кривых, заданных:}	
			\subsection*{\ \ \ \ \ \ a) Уравнениями в прямоугольной системе координат \ $y = 3 + \cosh \left(x\right), \ 0 \le x \le 1$}
				Если кривая задана в прямоугольной системе координат, уравнением $y = f(x)$ где $x \in [a, b]$, то ее длинанаходится по формуле: $L = \int_a^b \sqrt{1 + \left(f'\left(x\right)\right)^2} \dif x$. \\Используя данную формулу, вычислим длину кривой:
				\begin{multline*}
					L = \int_0^1 \sqrt{1 + \left(y'\right)^2} \dif x = \int_0^1 \sqrt{1 + \sinh^2 x} \dif x \\\textrm{Перейдем от гиперболического синуса к экспоненте, чтобы получить табличные интегралы:}\\ L = \int_0^1 \sqrt{1 + \left(\dfrac{e^x - e^{-x}}2\right)^2} \dif x = \dfrac 1 2 \int_0^1 \sqrt{4 + \left(e^{2x} - 2e^x \ e^{-x} + e^{-2x}\right)} \dif x = \\= \dfrac 1 2 \int_0^1 \dfrac{1}{e^x}\sqrt{4e^{2x} + e^{4x} - 2e^{2x} + 1} \dif x = \dfrac 1 2 \int_0^1 \dfrac{1}{e^x} \sqrt{2e^{2x} + e^{4x} + 1} \dif x = \dfrac 1 2 \int_0^1 \dfrac{1}{e^x} \sqrt{\left(e^{2x} + 1\right)^2} \dif x =\\= \dfrac 1 2 \int_0^1 \dfrac{1}{e^x}\left(e^{2x} + 1\right) \dif x = \dfrac 1 2 \int_0^1 e^x \dif x + \dfrac 1 2 \int_0^1 \dfrac{\dif x}{e^x} = \dfrac 1 2 \left(e^x\right)|_0^1 + \dfrac 1 2 \int_0^1 \dfrac{\dif x}{e^x} = \dfrac 1 2 \left(e - 1\right) + \dfrac 1 2 \int_0^1 \dfrac{\dif x}{e^x}
				\end{multline*}
				Найдем вторую часть интеграла  помощью замены:
				\begin{multline*}
					\treplace{t = e^x}{\dif t = e^x \dif x}{\dif x = \dfrac{\dif t}{e^x} = \dfrac{\dif t}t}, \ \ \ \ \ \begin{array}{l|l} x & t \\ -& - \\ 0 & 1 \\ 1 & e\end{array}\ \ \ \ => \dfrac 1 2 \left(e - 1\right) + \dfrac 1 2 \int_0^1 \dfrac{\dif x}{e^x} = \dfrac 1 2 \left(e - 1\right) + \dfrac 1 2 \int_1^e \dfrac{\dif t}{t^2} =\\= \dfrac 1 2 \left(e - 1\right) - \dfrac 1 2 \left.\left(\dfrac 1 t\right)\right|_1^e = \dfrac 1 2 \left(e - 1 - \dfrac 1 e + 1\right) = \dfrac{e^2 - 1}{2e}
				\end{multline*}
			
			\subsection*{\ \ \ \ \ \ б) Параметрически \ $\left\{ \begin{aligned} x = 4 \cos^3 t \\ y = 4 \sin^3 t \\ \end{aligned}\right. , \ \dfrac{\pi}6 \le t \le \dfrac{\pi}4$}
				Если кривая задана параметрическими уравнениями $\left\{ \begin{aligned} x = \phi\left(t\right) \\ y = \xi\left(t\right) \\ \end{aligned}\right. , \ \alpha \le t \le \beta$, где $\alpha < \beta$, то ее длина находится по формуле: $L = \int_\alpha^\beta \sqrt{\left(\phi ' \left(t\right)\right)^2 + \left(\xi ' \left(t\right)\right)^2} \dif t$. Здесь, естественно, предполагается, что функции $\phi\left(t\right), \xi\left(t\right)$ и их производные непрерывны на промежутке $[\alpha,  \beta]$ \\Используя данную формулу, вычислим длину кривой:
				\begin{multline*}
					L = \int_{\frac{\pi}6}^{\frac{\pi}4} \sqrt{\left(x'(t)\right)^2 + \left(y'(t)\right)^2} \dif t  = \int_{\frac{\pi}6}^{\frac{\pi}4} \sqrt{\left(-12 \cos^2 t \ \sin t\right)^2 + \left(12 \sin^2 t \ \cos t\right)^2} \dif t =\\= 12 \int_{\frac{\pi}6}^{\frac{\pi}4} \sqrt{\cos^4 t \ \sin^2 t + \sin^4 t \ \cos^2 t} \dif t = 12 \int_{\frac{\pi}6}^{\frac{\pi}4} \sin t \ \cos t \sqrt{\cos^2 t + \sin^2 t} \dif t = 12 \int_{\frac{\pi}6}^{\frac{\pi}4} \sin t \ \cos t \dif t\\\dreplace{u = sin t}{\dif u = \cos t \dif t}, \ \ \ \ \ \begin{array}{l|l} t & u \\ --& --- \\ \pi/6 & 1/2 \\ \pi/4 & \sqrt 2 / 2\end{array}\ \ \ \ => 12 \int_{\frac{\pi}6}^{\frac{\pi}4} \sin t \ \cos t \dif t = 12 \int_{\frac 1 2}^{\frac{\sqrt 2}2} u \dif u =\\= 6 \left.u^2\right|_{\frac 1 2}^{\frac{\sqrt 2}2} = 6\left(\left(\frac{\sqrt 2}2\right)^2 - \frac 1 4\right) = \dfrac 6 4 = \dfrac 3 2
				\end{multline*}
			
			\subsection*{\ \ \ \ \ \ в) В полярных координатах \ $\rho = 5 e ^{5\phi/12}, \ 0 \le \phi \le \dfrac{\pi}3$}
				Если кривая задана уравнением в полярных координатах $\rho = \rho\left(\phi\right), \ \alpha \le \phi \le \beta$, Причем, функция $\rho = \rho\left(\phi\right)$ и ее производная непрерывны на промежутке $[\alpha,  \beta]$, то ее длина находится по формуле: $L = \int_\alpha^\beta \sqrt{\rho^2 \left(\phi\right) + \left(\rho' \left(\phi\right)\right)^2} \dif \phi$ \\Используя данную формулу, вычислим длину кривой:
				\begin{multline*}
					L = \int_0^\frac{\pi}3 \sqrt{\rho^2 \left(\phi\right) + \left(\rho' \left(\phi\right)\right)^2} \dif \phi = \int_0^\frac{\pi}3 \sqrt{25 e^{5\phi/6} + \left(\dfrac{25}{12} \ e ^{5\phi/12}\right)^2} \dif \phi = \int_0^\frac{\pi}3 \sqrt{25 e^{5\phi/6} +  \dfrac{25^2}{144} \ e ^{5\phi/6}} \dif \phi =\\= 5 \int_0^\frac{\pi}3 \sqrt{e^{5\phi/6} + \dfrac{25}{144} \ e ^{5\phi/6}} \dif \phi = \dfrac 5{12} \int_0^\frac{\pi}3 e^{5\phi/12}\sqrt{12^2 + 5^2} \dif \phi = \dfrac{13*5}{12} \int_0^\frac{\pi}3 e^{5\phi/12} \dif \phi = \dfrac{65}{12} * \dfrac{12}{5} \left(e^{5\phi/12}\right)|_0^\frac{\pi}3 =\\= 13 \left(e^{5*\pi/12*3} - e^{5*0/12}\right) = 13\left(e^{5\pi/36} - 1\right)
				\end{multline*}
				
		\section{Вычислить объем тела, образованного вращением фигуры, \\ограниченной графиками функций (Вокруг оси Oy):}
			\subsection*{\ \ \ \ \ \ $y = \arccos\left(\dfrac x 3\right), \ y = \arccos x, \ y = 0$}
				Если объем V тела существует и $S = S\left(x\right), \ 0 \le a \le x \le b$, есть площадь сечения тела плоскостью, перпендикулярной к оси Ox в точке x, то $V = \int_a^b S\left(x\right) \dif x$. Добавим к этому условие, что криволинейная трапеция вращается вокруг оси Оу: $V_y = 2\pi \int_a^b x \ |f\left(x\right)| \dif x \ \ \ \ $ Наше тело имеет выколотую сердцевину, поэтому найдем полный объем и объем внутренней части, после чего вычтем второе из первого. Используя данную формулу, вычислим объем четверти тела вращения(всилу симметричности фигуры):
				\begin{figure}
					\centering
					\includegraphics{task_6}
				\end{figure}
				\begin{multline*}
					V = \dfrac 1 4 * 2\pi \int_0^3 \left|x \ \arccos{\left(\dfrac x 3\right)}\right| \dif x - \dfrac 1 4 * 2\pi \int_0^1 \left|x \ \arccos x \right|\dif x = \dfrac{\pi}2 \left(\left|\int_0^3 x \ \arccos{\left(\dfrac x 3\right)} \dif x\right| - \left|\int_0^1 x \ \arccos x \dif x\right|\right)\\ \int_0^3 x \ \arccos{\left(\dfrac x 3\right)} \dif x = (1), \ \int_0^1 x \ \arccos x \dif x = (2)
				\end{multline*}
				Под знаком интеграла стоит произведение полинома на трансцендентную функцию - это типовая ситуация. Избавляемся от трансцендентности, интегрируя по частям, где за u берем эту функцию, ее производная будет уже алгебраической функцией:
				\begin{multline*}
					\freplace{u = \arccos{\left(\dfrac x 3\right)}}{\dif u = -\dfrac 1 3 \dfrac{\dif x}{\sqrt{1 - \left(\dfrac x 3\right)^2}} =  - \dfrac{\dif x}{\sqrt{9 - x^2}}}{v = \dfrac{x^2}2}{\dif v =  x \ \dif x} =>\\ (1) = \left.\left(\dfrac{x^2}2 \arccos{\left(\dfrac x 3\right)}\right)\right|_0^3 - \int_0^3 - \dfrac{x^2}{2\sqrt{9 - x^2}} \dif x = \dfrac{3^2}2 \arccos{\left(1\right)} - \dfrac 1 2 \int_0^3 \dfrac{-x^2}{\sqrt{9 - x^2}} = \dfrac 1 2 \int_0^3 \dfrac{x^2}{\sqrt{9 - x^2}} \dif x\\\textrm{Выделим целую часть и снова проинтегрируем по частям:}\\- \dfrac 1 2 \int_0^3 \left(\dfrac{9 - x^2}{\sqrt{9 - x^2}}  - \dfrac{9}{\sqrt{9 - x^2}}\right)\dif x = \dfrac 9 2 \int_0^3 \dfrac{\dif x}{\sqrt{9 - x^2}}  - \dfrac 1 2 \int_0^3 \sqrt{9 - x^2}\dif x = \dfrac 9 2 \left.\arcsin{\left(\dfrac x 3\right)}\right|_0^3 - \dfrac 1 2 \int_0^3 \sqrt{9 - x^2}\dif x =\\= \dfrac 9 2\left(\arcsin 1 - \arcsin 0\right) - \dfrac 1 2 \int_0^3 \sqrt{9 - x^2}\dif x = \dfrac{9\pi}4 - \dfrac 1 2 \int_0^3 \sqrt{9 - x^2}\dif x\\\textrm{Интегрируем оставшийся интеграл по частям:} \ I = \int_0^3 \sqrt{9 - x^2}\dif x\\\freplace{u = \sqrt{9 - x^2}}{\dif u = -\dfrac{x \dif x}{\sqrt{9 - x^2}}}{v = x}{\dif v = \dif x} => I = \left.\left(x\sqrt{9 - x^2}\right)\right|_0^3 + \int_0^3 \dfrac{x^2}{\sqrt{9 - x^2}} \dif x =\\= -\int_0^3 \dfrac{-x^2}{\sqrt{9 - x^2}} \dif x = -\int_0^3 \left( \dfrac{9 - x^2}{\sqrt{9 - x^2}} - \dfrac{9}{\sqrt{9 - x^2}}\right)\dif x = -\int_0^3 \sqrt{9 - x^2}\dif x + 9 \int_0^3 \dfrac{\dif x}{\sqrt{9 - x^2}} =\\= 9 \left.\left(\arcsin{\left(\dfrac x 3 \right)}\right)\right|_0^3 -\int_0^3 \sqrt{9 - x^2}\dif x = -\int_0^3 \sqrt{9 - x^2}\dif x + 9 \left(\arcsin{\left(1\right)} - \arcsin{\left(0\right)}\right) = -I + \dfrac{9 \pi}{2} =>\\ 2I = \dfrac{9 \pi}{2} => I = \dfrac{9 \pi}{4}\ \ \ \ \ \ (1) = \dfrac{9\pi}{4} - \dfrac 1 2 I = \dfrac{9 \pi}8\\\\ \textrm{Теперь аналогично раскроем второй интеграл:}\\
					\freplace{u = \arccos{\left(x\right)}}{\dif u = -\dfrac{\dif x}{\sqrt{1 - x^2}}}{v = \dfrac{x^2}2}{\dif v =  x \ \dif x} => (2) = \dfrac 1 2 \int_0^1 \dfrac{x^2}{\sqrt{1 - x^2}} \dif x =\\= - \dfrac 1 2 \int_0^1 \left(\dfrac{1 - x^2}{\sqrt{1 - x^2}}  - \dfrac{1}{\sqrt{1 - x^2}}\right)\dif x = \dfrac 1 2 \int_0^1 \dfrac{\dif x}{\sqrt{1 - x^2}}  - \dfrac 1 2 \int_0^1 \sqrt{1 - x^2}\dif x = \dfrac{\pi}4 - \dfrac 1 2 \int_0^1 \sqrt{1 - x^2}\dif x = \dfrac{\pi}{4} - \dfrac 1 2 I\\  \freplace{u = \sqrt{1 - x^2}}{\dif u = -\dfrac{x \dif x}{\sqrt{1 - x^2}}}{v = x}{\dif v = \dif x} =>\\ I = \left.\left(x\sqrt{1 - x^2}\right)\right|_0^1 + \int_0^1 \dfrac{x^2}{\sqrt{1 - x^2}} \dif x =  -\int_0^1 \left(1 - \dfrac{1}{\sqrt{1 - x^2}}\right)\dif x = \left.\left(\arcsin{x}\right)\right|_0^1 -\int_0^1 \sqrt{1 - x^2}\dif x =\\= -I + \dfrac{\pi}{2} =>\ \ \ \  2I = \dfrac{\pi}{2} => I = \dfrac{\pi}{4}\ \ \ \ \ \ \ \ (2) = \dfrac{\pi}{4} - \dfrac 1 2 I = \dfrac{\pi}8\\ \textrm{Вернемся к исходному интегралу:}\ \ \ \ V = \dfrac{\pi}2 \left((1) - (2)\right) = \dfrac{\pi}2 \left(\dfrac{9\pi}8 - \dfrac{\pi}8\right) = \dfrac{\pi^2}2
				\end{multline*}
\end{document}









